\documentclass[]{article}
\usepackage{lmodern}
\usepackage{amssymb,amsmath}
\usepackage{ifxetex,ifluatex}
\usepackage{fixltx2e} % provides \textsubscript
\ifnum 0\ifxetex 1\fi\ifluatex 1\fi=0 % if pdftex
  \usepackage[T1]{fontenc}
  \usepackage[utf8]{inputenc}
\else % if luatex or xelatex
  \ifxetex
    \usepackage{mathspec}
  \else
    \usepackage{fontspec}
  \fi
  \defaultfontfeatures{Ligatures=TeX,Scale=MatchLowercase}
\fi
% use upquote if available, for straight quotes in verbatim environments
\IfFileExists{upquote.sty}{\usepackage{upquote}}{}
% use microtype if available
\IfFileExists{microtype.sty}{%
\usepackage{microtype}
\UseMicrotypeSet[protrusion]{basicmath} % disable protrusion for tt fonts
}{}
\usepackage[margin=1in]{geometry}
\usepackage{hyperref}
\hypersetup{unicode=true,
            pdftitle={Proyecto TID},
            pdfauthor={Alejandro Campoy Nieves; Gema Correa Fernández; Luis Gallego Quero; Jonathan Martín Valera; Andrea Morales Garzón},
            pdfborder={0 0 0},
            breaklinks=true}
\urlstyle{same}  % don't use monospace font for urls
\usepackage{color}
\usepackage{fancyvrb}
\newcommand{\VerbBar}{|}
\newcommand{\VERB}{\Verb[commandchars=\\\{\}]}
\DefineVerbatimEnvironment{Highlighting}{Verbatim}{commandchars=\\\{\}}
% Add ',fontsize=\small' for more characters per line
\usepackage{framed}
\definecolor{shadecolor}{RGB}{248,248,248}
\newenvironment{Shaded}{\begin{snugshade}}{\end{snugshade}}
\newcommand{\KeywordTok}[1]{\textcolor[rgb]{0.13,0.29,0.53}{\textbf{#1}}}
\newcommand{\DataTypeTok}[1]{\textcolor[rgb]{0.13,0.29,0.53}{#1}}
\newcommand{\DecValTok}[1]{\textcolor[rgb]{0.00,0.00,0.81}{#1}}
\newcommand{\BaseNTok}[1]{\textcolor[rgb]{0.00,0.00,0.81}{#1}}
\newcommand{\FloatTok}[1]{\textcolor[rgb]{0.00,0.00,0.81}{#1}}
\newcommand{\ConstantTok}[1]{\textcolor[rgb]{0.00,0.00,0.00}{#1}}
\newcommand{\CharTok}[1]{\textcolor[rgb]{0.31,0.60,0.02}{#1}}
\newcommand{\SpecialCharTok}[1]{\textcolor[rgb]{0.00,0.00,0.00}{#1}}
\newcommand{\StringTok}[1]{\textcolor[rgb]{0.31,0.60,0.02}{#1}}
\newcommand{\VerbatimStringTok}[1]{\textcolor[rgb]{0.31,0.60,0.02}{#1}}
\newcommand{\SpecialStringTok}[1]{\textcolor[rgb]{0.31,0.60,0.02}{#1}}
\newcommand{\ImportTok}[1]{#1}
\newcommand{\CommentTok}[1]{\textcolor[rgb]{0.56,0.35,0.01}{\textit{#1}}}
\newcommand{\DocumentationTok}[1]{\textcolor[rgb]{0.56,0.35,0.01}{\textbf{\textit{#1}}}}
\newcommand{\AnnotationTok}[1]{\textcolor[rgb]{0.56,0.35,0.01}{\textbf{\textit{#1}}}}
\newcommand{\CommentVarTok}[1]{\textcolor[rgb]{0.56,0.35,0.01}{\textbf{\textit{#1}}}}
\newcommand{\OtherTok}[1]{\textcolor[rgb]{0.56,0.35,0.01}{#1}}
\newcommand{\FunctionTok}[1]{\textcolor[rgb]{0.00,0.00,0.00}{#1}}
\newcommand{\VariableTok}[1]{\textcolor[rgb]{0.00,0.00,0.00}{#1}}
\newcommand{\ControlFlowTok}[1]{\textcolor[rgb]{0.13,0.29,0.53}{\textbf{#1}}}
\newcommand{\OperatorTok}[1]{\textcolor[rgb]{0.81,0.36,0.00}{\textbf{#1}}}
\newcommand{\BuiltInTok}[1]{#1}
\newcommand{\ExtensionTok}[1]{#1}
\newcommand{\PreprocessorTok}[1]{\textcolor[rgb]{0.56,0.35,0.01}{\textit{#1}}}
\newcommand{\AttributeTok}[1]{\textcolor[rgb]{0.77,0.63,0.00}{#1}}
\newcommand{\RegionMarkerTok}[1]{#1}
\newcommand{\InformationTok}[1]{\textcolor[rgb]{0.56,0.35,0.01}{\textbf{\textit{#1}}}}
\newcommand{\WarningTok}[1]{\textcolor[rgb]{0.56,0.35,0.01}{\textbf{\textit{#1}}}}
\newcommand{\AlertTok}[1]{\textcolor[rgb]{0.94,0.16,0.16}{#1}}
\newcommand{\ErrorTok}[1]{\textcolor[rgb]{0.64,0.00,0.00}{\textbf{#1}}}
\newcommand{\NormalTok}[1]{#1}
\usepackage{longtable,booktabs}
\usepackage{graphicx,grffile}
\makeatletter
\def\maxwidth{\ifdim\Gin@nat@width>\linewidth\linewidth\else\Gin@nat@width\fi}
\def\maxheight{\ifdim\Gin@nat@height>\textheight\textheight\else\Gin@nat@height\fi}
\makeatother
% Scale images if necessary, so that they will not overflow the page
% margins by default, and it is still possible to overwrite the defaults
% using explicit options in \includegraphics[width, height, ...]{}
\setkeys{Gin}{width=\maxwidth,height=\maxheight,keepaspectratio}
\IfFileExists{parskip.sty}{%
\usepackage{parskip}
}{% else
\setlength{\parindent}{0pt}
\setlength{\parskip}{6pt plus 2pt minus 1pt}
}
\setlength{\emergencystretch}{3em}  % prevent overfull lines
\providecommand{\tightlist}{%
  \setlength{\itemsep}{0pt}\setlength{\parskip}{0pt}}
\setcounter{secnumdepth}{0}
% Redefines (sub)paragraphs to behave more like sections
\ifx\paragraph\undefined\else
\let\oldparagraph\paragraph
\renewcommand{\paragraph}[1]{\oldparagraph{#1}\mbox{}}
\fi
\ifx\subparagraph\undefined\else
\let\oldsubparagraph\subparagraph
\renewcommand{\subparagraph}[1]{\oldsubparagraph{#1}\mbox{}}
\fi

%%% Use protect on footnotes to avoid problems with footnotes in titles
\let\rmarkdownfootnote\footnote%
\def\footnote{\protect\rmarkdownfootnote}

%%% Change title format to be more compact
\usepackage{titling}

% Create subtitle command for use in maketitle
\newcommand{\subtitle}[1]{
  \posttitle{
    \begin{center}\large#1\end{center}
    }
}

\setlength{\droptitle}{-2em}

  \title{Proyecto TID}
    \pretitle{\vspace{\droptitle}\centering\huge}
  \posttitle{\par}
    \author{Alejandro Campoy Nieves \\ Gema Correa Fernández \\ Luis Gallego Quero \\ Jonathan Martín Valera \\ Andrea Morales Garzón}
    \preauthor{\centering\large\emph}
  \postauthor{\par}
      \predate{\centering\large\emph}
  \postdate{\par}
    \date{14 de noviembre de 2018}


\begin{document}
\maketitle

\newpage 

\tableofcontents

\newpage 

\section{1. Comprender el problema a
resolver}\label{comprender-el-problema-a-resolver}

Para la realización y aplicación de las técnicas explicadas a lo largo
del curso, hemos seleccionado un \emph{dataset} proporcionado por
\href{https://archive.ics.uci.edu/ml/index.php}{\emph{UCI Machine
Learning Repository}}. En concreto, hemos escogido
\href{https://archive.ics.uci.edu/ml/datasets/Drug+Review+Dataset+\%28Druglib.com\%29}{\textbf{Drug
Review Dataset}}, una exhaustiva base de datos de medicamentos
organizada por relevancia para medicamentos específicos. El conjunto de
datos proporciona revisiones de pacientes sobre medicamentos específicos
junto con las condiciones relacionadas. Además, las revisiones se
agrupan en informes sobre tres aspectos: beneficios, efectos secundarios
y comentarios generales. De igual modo, las calificaciones están
disponibles con respecto a la satisfacción general, así como una
calificación de efectos secundarios de 5 pasos y una calificación de
eficacia de 5 pasos. Los datos se obtuvieron rastreando los sitios de
revisión farmacéutica en línea.

\begin{longtable}[]{@{}llllll@{}}
\toprule
\begin{minipage}[b]{0.18\columnwidth}\raggedright\strut
DataSet Characteristics:\strut
\end{minipage} & \begin{minipage}[b]{0.25\columnwidth}\raggedright\strut
Multivariate, Text\strut
\end{minipage} & \begin{minipage}[b]{0.15\columnwidth}\raggedright\strut
Number of Instances:\strut
\end{minipage} & \begin{minipage}[b]{0.04\columnwidth}\raggedright\strut
4143\strut
\end{minipage} & \begin{minipage}[b]{0.13\columnwidth}\raggedright\strut
Area:\strut
\end{minipage} & \begin{minipage}[b]{0.08\columnwidth}\raggedright\strut
N/A\strut
\end{minipage}\tabularnewline
\midrule
\endhead
\begin{minipage}[t]{0.18\columnwidth}\raggedright\strut
Attribute Characteristics:\strut
\end{minipage} & \begin{minipage}[t]{0.25\columnwidth}\raggedright\strut
Integer\strut
\end{minipage} & \begin{minipage}[t]{0.15\columnwidth}\raggedright\strut
Number of Attributes:\strut
\end{minipage} & \begin{minipage}[t]{0.04\columnwidth}\raggedright\strut
8\strut
\end{minipage} & \begin{minipage}[t]{0.13\columnwidth}\raggedright\strut
Date Donated\strut
\end{minipage} & \begin{minipage}[t]{0.08\columnwidth}\raggedright\strut
2018-10-02\strut
\end{minipage}\tabularnewline
\begin{minipage}[t]{0.18\columnwidth}\raggedright\strut
----------------------------\strut
\end{minipage} & \begin{minipage}[t]{0.25\columnwidth}\raggedright\strut
----------------------------------------\strut
\end{minipage} & \begin{minipage}[t]{0.15\columnwidth}\raggedright\strut
-----------------------\strut
\end{minipage} & \begin{minipage}[t]{0.04\columnwidth}\raggedright\strut
------\strut
\end{minipage} & \begin{minipage}[t]{0.13\columnwidth}\raggedright\strut
---------------------\strut
\end{minipage} & \begin{minipage}[t]{0.08\columnwidth}\raggedright\strut
------------\strut
\end{minipage}\tabularnewline
\begin{minipage}[t]{0.18\columnwidth}\raggedright\strut
Associated Tasks:\strut
\end{minipage} & \begin{minipage}[t]{0.25\columnwidth}\raggedright\strut
Classification, Regression, Clustering\strut
\end{minipage} & \begin{minipage}[t]{0.15\columnwidth}\raggedright\strut
Missing Values?\strut
\end{minipage} & \begin{minipage}[t]{0.04\columnwidth}\raggedright\strut
N/A\strut
\end{minipage} & \begin{minipage}[t]{0.13\columnwidth}\raggedright\strut
Number of Web Hits:\strut
\end{minipage} & \begin{minipage}[t]{0.08\columnwidth}\raggedright\strut
7001\strut
\end{minipage}\tabularnewline
\bottomrule
\end{longtable}

Los datos se dividen en un conjunto train (75\%) y otro conjunto test
(25\%) y se almacenan en dos archivos.tsv (tab-separated-values),
respectivamente. Los atributos que tenemos en este dataset son:

\begin{enumerate}
\def\labelenumi{\arabic{enumi}.}
\tightlist
\item
  \textbf{urlDrugName} (categorical): nombre de la droga
\item
  \textbf{condition} (categorical): nombre de la condición
\item
  \textbf{benefitsReview} (text): paciente sobre beneficios
\item
  \textbf{sideEffectsReview} (text): paciente sobre los efectos
  secundarios
\item
  \textbf{commentsReview} (text): comentario general del paciente
\item
  \textbf{rating} (numerical): clasificación de paciente de 10 estrellas
\item
  \textbf{sideEffects} (categorical): clasificación de 5 pasos de
  efectos secundarios
\item
  \textbf{effectiveness} (categorical): clasificación de efectividad de
  5 pasos
\end{enumerate}

\section{2. Prepocesamiento de datos}\label{prepocesamiento-de-datos}

Para poder analizar el dataset y realizar el prepocesamiento al mismo,
lo primero que se va hacer es leer tanto el conjunto de datos train como
de test. Primero, leeremos los datos con los que se va a entrenar y
luego los datos test.

\subsection{2.1. Lectura de datos}\label{lectura-de-datos}

A continuación, leemos nuestro dataset train y test:

\begin{Shaded}
\begin{Highlighting}[]
\CommentTok{# Lectura de datos train}
\NormalTok{datos_train <-}\StringTok{ }\KeywordTok{read.table}\NormalTok{(}\StringTok{"datos/drugLibTrain_raw.tsv"}\NormalTok{, }\DataTypeTok{sep=}\StringTok{"}\CharTok{\textbackslash{}t}\StringTok{"}\NormalTok{, }\DataTypeTok{comment.char=}\StringTok{""}\NormalTok{, }
                          \DataTypeTok{quote =} \StringTok{"}\CharTok{\textbackslash{}"}\StringTok{"}\NormalTok{, }\DataTypeTok{header=}\OtherTok{TRUE}\NormalTok{)}
\KeywordTok{head}\NormalTok{(datos_train, }\DecValTok{5}\NormalTok{) }\CommentTok{# visualizar las 5 primeras filas}
\KeywordTok{summary}\NormalTok{(datos_train) }\CommentTok{# información sobre los datos}
\KeywordTok{View}\NormalTok{(datos_train)    }\CommentTok{# vista de la tabla }

\CommentTok{# Lectura de datos test}
\NormalTok{datos_test <-}\StringTok{ }\KeywordTok{read.table}\NormalTok{(}\StringTok{"./datos/drugLibTest_raw.tsv"}\NormalTok{, }\DataTypeTok{sep=}\StringTok{"}\CharTok{\textbackslash{}t}\StringTok{"}\NormalTok{, }\DataTypeTok{comment.char=}\StringTok{""}\NormalTok{, }
                         \DataTypeTok{quote =} \StringTok{"}\CharTok{\textbackslash{}"}\StringTok{"}\NormalTok{, }\DataTypeTok{header=}\OtherTok{TRUE}\NormalTok{)}
\KeywordTok{head}\NormalTok{(datos_test, }\DecValTok{5}\NormalTok{) }\CommentTok{# visualizar las 5 primeras filas}
\KeywordTok{summary}\NormalTok{(datos_test) }\CommentTok{# información sobre los datos}
\KeywordTok{View}\NormalTok{(datos_test)    }\CommentTok{# vista de la tabla }
\end{Highlighting}
\end{Shaded}

\subsection{2.2. Falta de datos, categorización, normalización,
reducción de
dimensionalidad.}\label{falta-de-datos-categorizacion-normalizacion-reduccion-de-dimensionalidad.}


\end{document}
